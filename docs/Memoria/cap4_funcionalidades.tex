\chapter{Funcionalidades Destacadas}


\section{Control de Usuarios}

\section{Reprodución de Vídeo}
	Desexase que a aplicación permita a reprodución dos videos contidos, mediante tecnics de
	streaming ou pseudo-streaming. Neste caso empregarase o pseudo-streaming polo sinxela que
	resulta esta implementación empregando as capacidades do tag $<video>$ de HTML5 en conxunto
	con un servidor HTTP como Apache ou o servidor para desenvolvemento de Django.

\section{Carga de Vídeo}
	Para a subida de vídeo empregarase un formulario HTML que empregue unha chamada POST de HTTP.
	Cando isto ocorra, o vídeo comezará a subirse ao servidor.
	
	É de especial importancia que no caso de que o vídeo teña un peso considerable e precise 
	duns cantos segundos para subirse á plataforma, o usuario poida coñecer de forma gráfica
	o avance deste proceso.
	
	Con este fin, crease un sistema de notificación de progreso baseado no 
	django-progressbarupload \cite{django-progressbarupload}.

	Este sistema crea no lado servidor basease na interface de Django TemporaryFileUploadHandler
	\cite{TemporaryFileUploadHandler}.
	
	Na función de inicio (handle\_raw\_input) crea unha entrada na Cache de Django, que almacena como 
	chave un número aleatorio e a IP do cliente que está a subir o vídeo, e como valor o tamaño
	do ficheiro e o porcentaxe de este que xa está subido.Esta entrada será actualizada cada vez 
	que o servido reciba un novo anaco de vídeo (mediante a función receive\_data\_chunk). 
	
	Por outra parte, para que visualmente o cliente poida ver o avance da subida mediante unha
	barra de progreso, crease unha función asíncrona en javascript (Tecnoloxía AJAX), que 
	periodicamente consulta ao servidor para obter o valor da cache que indica a porcentaxe de
	subida do vídeo, e unha vez obtido, actualiza a barra de progreso para mostralo. Todo isto
	ten lugar no navegador mentres este inda está a subir o arquivo de vídeo.

	Unha vez que a subida se completa, o POST é manexado pola vista UploadView, que se todos
	os datos do formulario son correctos, encargase de crear un modelo VideoModel. Como parte
	desta creación o vídeo pasa do directorio temporal no que foi almacenado (baixo linux por 
	defecto é /tmp) ao un directorio calculado pola función get\_valid\_filename. Esta función
	pásaselle ao modelo como parte do seu campo ''video'' do tipo FileField.
	
\section{Análise de Vídeo}
	Para a análise de vídeo

\section{Detección de Obxectos}

\section{Traxectorias}

\section{Comportamento anormal}
