\chapter{Fundamentos Teóricos e Conceptos Previos}

En este capítulo se exponen los fundamentos teóricos en los que se basa el proyecto o que se
utilizan en él.
Normalmente cada proyecto trata de una temática, de la que hay una serie de conocimientos
teóricos relacionados y que deben incluirse en la memoria del proyecto, para aportarle el carácter
científico que conlleva todo trabajo académico.
Se requiere una explicación detallada de todo lo necesario, para dar una visión profunda de lo
que debe conocerse para afrontar la realización del proyecto. Por una parte sirve como exposición
de los elementos científicos en los que se basa el proyecto, pero a la vez, como explicación de lo
que se ha estudiado para la elaboración del mismo.

  \section{Programación Web}

  \section{OpenCV}

  \section{Análise do Comportamento}
  
\chapter{Análise de antecedentes e alternativas}
	Se trata de realizar un estudio de alternativas o “estado del arte” o un análisis comparativo
	de alternativas.

	Se exponen las diferentes alternativas que se han evaluado o que se consideran de interés, a
	lo realizado en el proyecto. Fundamentalmente se trata de otras herramientas existentes 
	que realizan algo similar, sean o no comerciales, o de prototipos de investigación relacionados,
	o de estudios que tratan aspectos similares.

	Buscar por internet produtos que fagan algo similar...