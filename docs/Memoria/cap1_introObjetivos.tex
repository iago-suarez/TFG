\chapter{Introdución e Obxectivos}

O seguimento de obxectos é o proceso de estimar no tempo a localización de un ou máis 
obxectos en movemento empregando as imaxes captadas por una cámara. A crecente mellora 
na potencia de cálculo dos procesadores actuais, xunto coa dixitalización dos sensores 
de imaxe propiciou dende comezos de século a aparición de novos algoritmos de análise 
que aportan cuantiosas melloras a este campo.

Neste aspecto, o Grupo de Visión Artificial e Recoñecemento de Patróns (VARPA)
da UDC leva anos investigando para aportar á comunidade científica os seus propios algoritmos
e desenvolver novas aplicación que empreguen estes algoritmos para detección de persoas,
vehículos, ou calquera outro obxecto susceptible de seres estudado. En concreto, o
grupo posúe ferramentas que permiten o seguimento en zoas transitadas nas que poden 
aparecer multitude de obxectos a seguir simultaneamente. 

Co fin de achegar estes métodos de análise á súa aplicación final, proponse dende o 
laboratorio a construción dunha web, que sexa capaz de reproducir vídeos, e sobre eles
mostrar distintas capas con información de alto nivel, como pode ser a resultante de
detectar obxectos, medir o seu grao de anormalidade, a súa velocidade, etc. 

Seleccionase unha arquitectura web xa que a diferencia das arquitecturas de escritorio,
proporciona ás persoas que acceden á web independencia do Sistema Operativo empregado e
dispoñibilidade dende calquera lugar con acceso a rede, evitando así as dificultades 
asociadas coa instalación ou actualización da aplicación. 
