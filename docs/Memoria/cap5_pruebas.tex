\chapter{Probas Realizadas}

Á hora de deseñar probas é importante abarcar a maior parte do código posible, neste proxecto isto
foi todo un reto, pois o alto nivel de integración dificulta enormemente a realización das probas.
Pese a todo, logrouse probar tanto o código realizado en Python-Django así como o código da capa 
cliente en javascript, empregando para elo distintos modelos e bibliotecas de probas que vemos a
continuación.

\section{Probas Unitarias}
  \subsection{Probas Automáticas}
  Estaría ben!

\section{Probas de Integración}
    \subsection{Probas funcionais Selenium}

    \subsection{Probas Web con Javascript}
        As probas da capa web en escrita en javascript apoiaranse no framework Qunit de 
        jQuery que proporciona un xeito sinxelo de crear probas unitarias sempre e cando o código 
        javascript esté convintemente separado do HTML que forma a vista da capa web.
        
        Como resulta lóxico, estas probas estarán escritas en javascript e almacenadas no directorio
        do proxecto /src/static/site/tests, podendo executarse de dous xeitos diferentes: Ou ben 
        como unha páxina web pertencente á aplicación (isto favorece o desenvolvemento áxil), ou ben
        como unha proba das realizadas polo comando python manage.py test.
        
        Para poder executar un código javascript dende a execución común dos tests da aplicación, 
        precisamos un lanzador ou runner que lance estes tests contra algún navegador de liña de 
        comandos, neste caso a opción seleccionada foi a combinación do paquete django.js (v0.8.1)
        en combinación co navegador de liña de comandos phantomJS. 
        
        Django.js é un conxunto de utilidades que permiten a integración de código javascript e 
        Django, e en concreto neste proxecto empréganse aquelas que teñen que ver co testing de
        aplicacións\ref{DjangojsTestTools}, mais en concreto QUnitSuite e JsTemplateTestCase para a
        execución dos tests como parte dos test da aplicación (sen parte servidor) e QUnitView 
        para a execución a modo de páxina web.
        
        Po outro lado PhantomJS é un navegador WebKit de liña de comandos, cunha API Javascript que
        da soporte rápido e nativo para varios estándares web que resultan moi do noso interese,
        como son a manipulación DOM, os selectores CSS, JSON, Canvas e SVG. PhantomJS será chamado
        implícitamente polo runner de Django.js cando se executen os test, mentres que no caso da 
        vista QUnitView os tests executaranse directamente no navegador que realice a petición.

\section{Probas de Sistema}

\section{Probas de Aceptación}

\chapter{Calidade}
	Os parámetros de calidade empregados para a codificación do código fonte son:
		JavaScript Style Guide and Coding Conventions\cite{javascript-style-guide}
		JavaScript Best Practices:\cite{javascript-best-practices}
		PEP8 Style Guide for Python Code: \cite{pepe8-style-guide}