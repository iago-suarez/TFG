\chapter{Metodoloxía seguida no desenvolvemento de Proxecto}

Os diferentes obxectivos do proxecto abordáronse seguindo a Metodoloxía SCRUM, adaptada a
un proxecto de un solo Developer.

Esta metodoloxía áxil tamén chamada melé pola súa inspiración no Rugby, permite un
desenvolvemento rápido en situacións de requisitos inestables. Apoiase no seu carácter 
iterativo e incremental, divindoo traballo a realizar en períodos de aproximadamente un mes
chamados Sprint's.

Para a realización deste traballo de fin de grao foi preciso adaptala, pois está pensada en 
principio para organizar equipos de entre 3 a 9 persoas (Team). Por outro lado, o marco de 
traballo planifica reunións diarias (Daily Scrum), ao supoñer que todos os membros do equipo 
traballan unha xornada laboral enteira entre cada unha destas reunións, o cal tampouco se dá 
no caso deste proxecto, xa que a dedicación será de determinadas horas nos momentos dispoñibles.

\section{Persoas}

Os tres papeis que se definen nesta metodoloxía \cite{la-guia-de-scrum} foron 
adaptados do seguinte xeito:
\begin{itemize}

	\item{\textbf{ProductOwner:}}
	O papel de ProductOwner, que define os requisitos da aplicación estivo representado 
	polo director de proxecto Brais Cancela, que participou na creación do Anteproxecto.
	En certos momentos o señor Cancela tamén desempeñou a función de membro do equipo, 
	posto que é foi autor do algoritmo de análise de vídeo. 

	\item{\textbf{ScrumMaster e Development Team:}}
	Ambos papeis leváronse a cabo polo autor, xa que carece de sentido definir ambas figuras
	nun equipo de unha soa persoa. De este xeito á par que se desenvolvía o proxecto, íase
	asegurando o cumprimento das regras de SCRUM.

\end{itemize}

\vspace{1cm}
\section{Reunións}
As reunións pola súa parte modifícanse do seguinte xeito:

\begin{itemize}
	\item{\textbf{Sprint Planning Meeting:}}
	Esta reunión mantén o mesmo formato que no SCRUM puro, xuntando ao autor co	ProductOwner 
	e concretando as tarefas do Product Backlog que se realizarán no seguinte Sprint, pasando
	por tanto a formar parte do Sprint Backlog.
	
	\item{\textbf{Daily Scrum:}}
	Dado que o equipo de Desenvolvemento e o ScrumMaster están conformados pola mesma persoa
	e que o número de horas diarias adicadas é moito menos ao dunha xornada laboral, considerouse
	oportuno substituír esta reunión diaria por unha reunión dúas veces á semana (Martes e Xoves 
	pola tarde normalmente). Na que se mostrase ao ProductOwner o avance do proxecto.
	
	\item{\textbf{Sprint Review:}}
	Esta reunión fusionase co Sprint Planning Meeting, xa que ao mesmo tempo valorase o traballo
	realizado no Sprint que remata e, en base a el, planificase a videira Iteración. 
	
	\item{\textbf{Sprint Retrospective:}}
	Pola súa parte, esta reunión toma un carácter unipersoal, pasando a ser unha valoración do
	propio autor sobre as persoas, relaciones, procesos e ferramentas implicadas no último Sprint.
	Nela avalíase os elementos con éxito e os suxeitos a melloras, creando un plan para implementar
	estas melloras na Videira iteración.
	
\end{itemize}


\section{Control de Versións con GitHub}

  Bla bla bla

  \subsection{Título}

  Bla bla bla

  \subsubsection{Título}

  Bla bla bla

\section{Integración Continua con TravisCI}
\section{Control da cobertura con Coveralls}
\section{Xestión de Incidencias e Control de Proxecto con YouTrack}
