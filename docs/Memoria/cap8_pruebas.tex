\chapter{Validación}

Á hora de deseñar probas é importante abarcar a maior parte do código posible, neste proxecto isto
foi todo un reto, pois o alto nivel de integración dificulta enormemente a realización das probas.
Pese a todo, logrouse probar tanto o código realizado en Python-Django así como o código da capa 
cliente en javascript, empregando para elo distintos modelos e bibliotecas de probas que vemos a
continuación.

\section{Probas Unitarias}

    As probas unitarias proban as funcionalidades mais básica do software. Executanse sempre no 
    ámbito de un só modulo para probar o correcto funcionamento de este, simulando se fose preciso 
    a súa interacción con outros módulos (este proceso chamase mocking).
    
    No ámbito da nosa aplicación as probas de cada módulo recollense nos ficheiros tests.py de cada
    un deles. Estas probas inclúen tanto comprobacións de funcións illadas, como a correcta xeración
    de algunha webs independentes do resto dos módulos.
    
    \subsection{Política de acceso ás páxinas web}
        Como resulta lóxico non todo o mundo pode acceder a tódalas páxinas da aplicación, algunhas
        delas están reservadas para o administrador, outras para o propio usuario logueado, unhas
        terceiras para calquera usuario logueado e por último hai páxinas de dominio público.
        
        É importante de cara a non cometer erros de seguridade, que este correcto comportamento sexa
        comprobado, así na táboa excel que se atopa no ficheiro docs/urlsMap.ods pódese observar con
        detalle que política de acceso segue cada unha das páxinas da aplicación segundo a súa URL.
        Foi a partir deste mapa de urls dende onde se elaborou a base dos test de unidade para 
        o acceso ás páxinas, que comprobar para cada unha das páxinas da aplicación que un usuario
        cos permisos adecuados poida acceder e que calquera dos demais reciba o erro axeitado.
        
        Pero só con probas unitarias non se pode asegurar a calidade dos software xa que moitos dos
        módulos están pensados para interactuar entre eles, polo que fanse necesarias as Probas de
        Integración.

    \subsection{Probas da Capa Web con Javascript}
        As probas da capa web en escrita en javascript apoiaranse no framework Qunit de 
        jQuery que proporciona un xeito sinxelo de crear probas unitarias sempre e cando o código 
        javascript esté convintemente separado do HTML que forma a vista da capa web.
        
        Como resulta lóxico, estas probas estarán escritas en javascript e almacenadas no directorio
        do proxecto /src/static/site/tests, podendo executarse de dous xeitos diferentes: Ou ben 
        como unha páxina web pertencente á aplicación (isto favorece o desenvolvemento áxil), ou ben
        como unha proba das realizadas polo comando python manage.py test.
        
        Para poder executar un código javascript dende a execución común dos tests da aplicación, 
        precisamos un lanzador ou runner que lance estes tests contra algún navegador de liña de 
        comandos, neste caso a opción seleccionada foi a combinación do paquete django.js (v0.8.1)
        en combinación co navegador de liña de comandos phantomJS. 
        
        Django.js é un conxunto de utilidades que permiten a integración de código javascript en 
        Django. Mais en concreto neste proxecto empréganse aquelas que teñen que ver co testing de
        aplicacións\ref{DjangojsTestTools}, destacando as clases QUnitSuite e PhantomJsRunner que se
        empregan para lanzar os tests como parte dos test da aplicación mediante a clase creada
        StaticJsTestCase que obtén os resultados do modo de páxina web co navegador de liña de 
        comandos, e tamén a clase QUnitView que é a peza central para a execución a neste modo de
        páxina web.
        
        Po outro lado PhantomJS é un navegador WebKit de liña de comandos, cunha API Javascript que
        da soporte rápido e nativo para varios estándares web que resultan moi do noso interese,
        como son a manipulación DOM, os selectores CSS, JSON, Canvas e SVG. PhantomJS será chamado
        implícitamente polo runner de Django.js cando se executen os test, mentres que no caso da 
        vista QUnitView os tests executaranse directamente no navegador que realice a petición.

\section{Probas de Integración}

    As probas de integración so aquelas que proban o funcionamento conxunto de varios módulos da 
    aplicación, realizanse tras o éxito das probas unitarias tamén sen que haxa interacción humana.
    
    No noso caso agruparemos as probas de integración nun paquete a parte, para evitar así 
    mesturalas coas probas unitarias de cada módulo. A estes efectos creamos a clase SeleniumAncowebTest
    que extende StaticLiveServerTestCase engadindo ademais os métodos login\_user(self, user, 
    password) e logout\_user(self, user) xa que todo-los tests que comproben outros módulos precisando
    de un usuario logueado consideranse tests de integración.

    \subsection{Probas Funcionais Selenium}

        As probas funcionais son aquelas nas que se lle dita ao sistema cales serán as saídas a unha 
        determinada serie de entradas co fin de comprobar que a funcionalidade é a correcta. No caso da
        nosa aplicación empregaremos tests funcionais para as probas de integración apoiandonos en Selenium.
        
        Selenium é un framework para a realización de probas funcionais que permite lanzar un navegador
        e indicar as accións a realizar sobre él xunto cunha serie de comprobacións para verificar que
        estas accións provocan na páxina web o efecto desexado. Resultan de especial relevancia na 
        programación web, xa que o seu functionamento asemellase moito ao que un humano faría para 
        comprobar o correcto funcionamento da web sendo por tanto moi intuitivo.
        
        %TODO Modificar si se testean mais cousas 
        No noso caso comprobanse con Selenium, o logueado de Usuarios, a subida de vídeos e o listado de
        vídeos.
    
\section{Probas de Sistema}

    %TODO 

\section{Probas de Aceptación}
    Por último están as probas de Aceptación que se fan co obxectivo de comprobar se o software cumpre
    coas expectativas que o cliente tiña de el. A estes efectos cada vez que se finalizaba unha 
    funcionalidade realizouse acorde coa metodoloxía unha proba completa por parte do titor Brais 
    Cancela que garantise que todo o implementado era acorde cos que se desexaba. Ocasionalmente o 
    proxecto tamén foi revisado polos outros titores aportandolle así un toque mais plural.
    

\chapter{Calidade}
	Os parámetros de calidade empregados para a codificación do código fonte son:
		JavaScript Style Guide and Coding Conventions\cite{javascript-style-guide}
		JavaScript Best Practices:\cite{javascript-best-practices}
		PEP8 Style Guide for Python Code: \cite{pepe8-style-guide}