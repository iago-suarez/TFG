\chapter{Apéndice}

Bla bla bla

\section{Lista de Acrónimos}

  Bla bla bla
\begin{itemize}
 \item {IDE}
 \item {BD}
\end{itemize}


\section{Manual de Usuario}

\section{Manual de referencias Técnicas}

\section{Notas acerca da Terminoloxía}
	% SCRUM
\begin{itemize}
	\item{Scrum Team}
	\item{Product Owner}
	\item{Development Team}
	\item{Scrum Master}
	\item{Spring}
	\item{Sprint Planning Meeting}
	\item{Sprint Goal}
	\item{Daily Scrum}
	\item{Sprint Review}
	\item{Sprint Retrospective}
	\item{Product Backlog}
	\item{Sprint Backlog}
	
	%Outros
	
\end{itemize}

  Se pueden escribir unas líneas en las que se justifique
  el motivo de la inclusión de términos en inglés, por ejemplo por formar parte de la
  jerga, ser de uso frecuente y estar completamente aceptados por comités científicos de
  la temática del proyecto, etc.; en otro caso deben buscarse traducciones adecuadas.
