\chapter{Apéndice}

\section{Lista de Acrónimos}

\begin{itemize}

    \item \textbf{AJAX:} Asynchronous JavaScript And XML (JavaScript asíncrono y XML)
    \item \textbf{ASP:} Active Server Pages
    \item \textbf{BD:} Base de Datos
    \item \textbf{CI:} Continuous Integration (integración continua)
    \item \textbf{CPU:} Central Processing Unit (unidade central de procesamento)
    \item \textbf{CSS:} Cascading Style Sheets (folla de estilo en cascada o CSS)
    \item \textbf{DNS:} Domain Name System o DNS (Sistema de Nomes de Dominio)
    \item \textbf{JSP} (de Spring): JavaServer Pages (páxinas servidas por Java)
    \item \textbf{JSON:} JavaScript Object Notation (notación de obxectos JavaScript )
    \item \textbf{DOM:} Document Object Model (modelo de obxectos do documento)
    \item \textbf{DTD:} Document Type Definition (definición de tipo de documento)
    \item \textbf{DV:} Digital Video (protocolo de vídeo dixital) 
    \item \textbf{GB:} Gigabyte
    \item \textbf{HTML:} HyperText Markup Language (linguaxe de marcas de hipertexto)
    \item \textbf{IDE:} Integrated Development Environment (Ambiente de desenvolvemento integrado)
    \item \textbf{ISO:} International Organization for Standardization (Organización Internacional de Normalización)
    \item \textbf{IP:} Internet Protocol (protocolo de internet)
    \item \textbf{JNI:} Java Native Interface (interface nativa para Java)
    \item \textbf{MPEG:} Moving Picture Experts Group 
    \item \textbf{MVC:} Modelo–vista–controlador
    \item \textbf{ORM:} Object-Relational Mapping (mapeador obxecto-relacional)
    \item \textbf{RAM:} Random Access Memory (memoria de acceso aleatorio)
    \item \textbf{RTPS:} Real Time Streaming Protocol (protocolo de fluxo en tempo real)
    \item \textbf{RTP:} Real-time Transport Protocol (protocolo de Transporte de Tempo real)
    \item \textbf{RTP Control Protocol:} RTP Control Protocol (protocolo de control para RTP)
    \item \textbf{SEO:} Search Engine Optimization (posicionamento en buscadores)
    \item \textbf{SGBD:} Sistema de Xestión de Bases de Datos
    \item \textbf{SSD:} Solid-State Drive (unidade de estado sólido)
    \item \textbf{UDC:} Universidade Da Coruña
    \item \textbf{USB:} Universal Serial Bus (bus universal en serie)
    \item \textbf{UI:} User Interface (interface de usuario)
    \item \textbf{URI:} Uniform Resource Identifier (identificador de recursos uniforme)
    \item \textbf{URL:} Uniform Resource Locator (localizador de recursos uniforme)
    \item \textbf{VARPA:} Grupo de Visión Artificial e Recoñecemento de Patróns
    \item \textbf{W3C:} World Wide Web Consortium
    \item \textbf{XML:} eXtensible Markup Language (linguaxe de marcas extensible)
    
\end{itemize}


\section{Manual de Usuario}
    Pódese consultar o manual de usuario no arquivo do proxecto\\
    ``docs/ManualDeUsuario/userManual.pdf''.

\section{Notas acerca da Terminoloxía}

    Aquí englóbanse os términos que non están oficialmente recoñecidos polo Dicionario da Real 
    Academia Galega, ben por ser términos provintes do inglés ou por ser específicos do mundo tecnolóxico:
\begin{itemize}
    \item \textbf{Adaptativo:} Adxectivo que define algo que se pode adaptar.
    \item \textbf{Assert:} Función típica nos tests de unidade que comproba unha determinada condición. 
    \item \textbf{Back-end:} Nunha aplicación web fai referencia á parte do programa que corre na máquina 
    servidor.
    \item \textbf{Buffer:} Memoria temporal de almacenamento temporal de información que permite transferir
    \item \textbf{Bug:} Erro no código dunha aplicación.
    \item \textbf{Daily Scrum:} Reunión de 15 minutos que se produce todo-los días a primeira hora.
    \item \textbf{Desenvolvemento colaborativo:} Desenvolvemento de un produto que se leva a cabo grazas á 
        colaboración de distintas persoas.
    \item \textbf{Developer:} Persoa que desenvolve unha aplicación informática.
    \item \textbf{Escalable:} Dise de un sistema, unha rede ou un proceso cando este ten a habilidade para
        reaccionar e adaptarse sen perder calidade nos servizos ofrecidos.
    \item \textbf{Frame:} Fotograma de vídeo.
    \item \textbf{Framework:} Conxunto estandarizado de conceptos, prácticas e criterios para enfocar unha
        problemática particular que serve como referencia para enfocar problemas de índole similar. 
        Normalmente componse de unha serie de librerías que en vez de ser chamadas dende o código da 
        aplicación que emprega este framework son elas as que chaman ao framework implementando o 
        patrón Inversión de Control (Inversion of Control en inglés, IoC).
    \item \textbf{Front-end:} Nunha aplicación web fai referencia á parte do programa interactúa cos usuarios.
    \item \textbf{Hosting:} É o servizo que ofrece aos usuarios de Internet un sistema para almacenar 
        información, imaxes, vídeo ou calquera contido accesible a través da web.    
    \item \textbf{Iterativo:} Adxectivo que describe un comportamento que se repite en varias iteracións. 
    \item \textbf{Implementar:} Poñer en funcionamento ou levar a cabo unha cousa determinada.    
    \item \textbf{Incremental:} Adxectivo que aplicado a unha metodoloxía quere dicir que en cada iteración
        se engade unha serie de novas funcionalidades. 
    \item \textbf{Iterar:} Repetir unha acción un número de veces. 
    \item \textbf{Layout:} Fai referencia ao esquema de distribución dos elementos dentro do deseño dunha 
        páxina web.
    \item \textbf{Linguaxe de scripting:} Linguaxe que se emprega para crear programas usualmente simples
        que polo regular almacenase en ficheiros de texto plano.
    \item \textbf{Loguear:} Identificar un usuario nunha aplicación.
        os datos entre unidades funcionais con características de transferencia diferentes.
    \item \textbf{Loseless:} Algoritmo de compresión sen perda.
    \item \textbf{Manexador (handler):} Obxecto encargado de xestionar unha determinada chamada.    
    \item \textbf{Mapeado (Mapping):} É o proceso de crear un mapa de algo, empregado entre dous sistemas ou
        dúas partes de un sistema significa trasladar a información do formato dun dos sistemas ao 
        formato do outro.
    \item \textbf{Multiplataforma:} Dise de unha aplicación capaz de correr sobre distintas plataformas como
        Mac OS, Linux e Windows por exemplo.
    \item \textbf{Parsing:} (do verbo inglés parse) Fai referencia ao proceso de separar unha frase en partes
        gramaticais, como suxeito, verbo, etc. Neste caso empregase cando se analizan documentos 
        para facer que un sistema comprenda o seu contido.
    \item \textbf{Plantilla(Template):} Esquema ou modelo empregado para crear unha serie de cousas 
        repetidamente.    
    \item \textbf{Portabilidade:} Característica dun programa, que lle permite ser executado en múltiples 
        tipos de sistemas operativos ou diferentes máquinas.
    \item \textbf{Posicionamento:} Proceso de colocar ou situar unha cousa na posición adecuada.
    \item \textbf{Product Backlog:} Lista de tarefas pendentes para a realización do produto.
    \item \textbf{Product Owner:} O product Owner representa a voz do cliente. Asegurase de que o equipo 
        traballe de forma adecuada dende o punto de vista do negocio, tamén é o encargado de 
        escribir e priorizar as historias de usuario no Product Backlog.
    \item \textbf{Responsive:} Coa capacidade de adaptarse a distintos tamaños de pantalla en diferentes 
        tipos de dispositivos.
    \item \textbf{Ratio:} Relación cuantificada entre dúas magnitudes que reflexa a súa proporción cun número
        entre 1 e 0.
    \item \textbf{Redirixir (redirect):} Proceso soportado pola técnica ``URL redirection'', que permite 
    dende un servidor enviar a un navegador a unha dirección diferente da que el requiriu.  
    \item \textbf{Relacional:} Almacenado seguindo o Modelo Relacional.
    \item \textbf{Renderizar:} proceso de interpretar uns esquemas para xerar un produto como unha páxina
        web, un deseño...
	\item \textbf{Scrum Master:} Persoa encargada de eliminar os obstáculos que impiden que o equipo alcance
        as súas metas no sprint. Non é o líder de equipo, xa que este é auto-organizado, se non que 
        actúa como protección entre o equipo e as posibles influenzas externas que o distraian. Tamén
        é o encargado de facer que as regras da metodoloxía SCRUM se cumpran.
    \item \textbf{Securizar:} Proceso de facer mais seguro un software. 
	\item \textbf{Sprint:} Iteración de aproximadamente un mes de duración. 
	\item \textbf{Sprint Planning Meeting:} Reunión que se produce ao inicio de cada un dos Sprints.	
	\item \textbf{Sprint Review:} Reunión que serve para revisar o traballo feito.
	\item \textbf{Sprint Retrospective:} Reunión que revisa o traballo realizado ao finalizar cada Sprint.
	\item \textbf{Sprint Backlog:} Lista de tarefas para un Sprint determinado.
    \item \textbf{Team:} Equipo de traballo sobre o que recae a responsabilidade de entregar o 
        produto. Normalmente confórmano de 3 a 9 persoas.
    \item \textbf{Thread:} Fío de execución de un computador.
    \item \textbf{Tracking:} Proceso de identificar os obxectos presentes nunha escena.
    \item \textbf{Transcodificar:} Pasar un arquivo de un formato de codificación a outro.
    \item \textbf{Usabilidade:} Dise de un sitio web ou software que son doados de empregar, porque facilitan 
        a lectura dos textos, descargan a información de xeito rápido e presentan unhas funcións e
        un menú sinxelo, polo que o usuario satisfai as súas consultas de xeito sinxelo e cómodo.
	
\end{itemize}
