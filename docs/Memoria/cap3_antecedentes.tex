\chapter{Análise de antecedentes e alternativas}
    En canto aos produtos existentes ate o momento que traballan no ámbito da detección de obxectos
    e a análise do comportamento, hai que diferenciar aqueles produtos da empresa
    privada que prometen grandes resultados pero dos cales debido á falta de medios non se pode 
    comprobar o seu correcto funcionamento, daqueles que pertencen ao mundo do software libre ou que
    están apoiados en documentos científicos e que si teñen unha reputación contrastada.
    
    
    \section{Software de carácter privativo}
    
        Do primeiro grupo pódese salientar toda unha serie de ferramentas que permiten o tratamento de 
        distintas señais de vídeo procedentes de varias cámaras sobre as que se executa unha análise que
        dependendo do producto ofrecenos uns datos ou outros. Algúns exemplos de estes productos son:
        
        \begin{itemize}
        \item \textbf{CyeWeb:}\cite{CyeWeb}
            É unha aplicación de escritorio para monitorización de cámaras de vixilancia que segundo 
            detallan na súa páxina web permite detección de movemento, conteo de obxectos, detección de
            aparcamentos en zoas non permitidas, detección de obxectos conflitivos...
            
        \item \textbf{Cerebrus Intelligent Video Analytics:}\cite{adventura-cerebrus-intelligent-video-analytics}
            Outra aplicación de escritorio que tamén obtén en directo o vídeo de varias cámaras de 
            vixilancia e mostra sobre este vídeo unha análise de alto nivel capaz de detectar intrusos,
            monitorizar de vehículos, detectar obxectos abandonados, contar persoas, etc.
        \end{itemize}
        
        En software privativo pero máis enfocados cara o tema da análise de comportamento e menos
        focalizados como programas de video-vixilancia temos outra serie de produtos como poden ser:
        
        \begin{itemize}
        \item  \textbf{Nolus:}\cite{nolus-human-behaviour}
            Dende esta empresa aseguran que o seu software é capaz de recoñecer distintos patróns de 
            comportamento en seres humanos e en animais.
        
        \item \textbf{WINanalyze:}\cite{WINanalyze-web-page}
            WINanalyze é unha aplicación de escritorio para a análise de vídeo procedente tanto de 
            cámaras en vivo como de ficheiros, que permite analizar os movementos realizados por un 
            obxecto do que se seguen distintos puntos de interese. Está baseado nunha serie de artigos
            científicos de comezos dos anos noventa sobre o seguimento de pixeles e puntos de interese
            que se poden consultar na web da empresa á que pertence\cite{mikromak-publications}.
        
        \item \textbf{Huygens Software - ObjectTracker:}\cite{Huygens-ObjectTracker}
            Huygens Software Suite é un conxunto de paquetes para o procesamento de imaxes.
            Esta suite centrada na análise de partículas inclúe funcionalidades como análise interactivo,
            visualización de volumes en 2D e 3D, funcionalidades para tratamento de imaxes procedentes
            de microscopio ou no tracking de obxectos mediante fluxo óptico: estudo das posicións,
            traxectorias, velocidades ...
            
        \end{itemize}

    \section{Software libre}
    \label{sec:video-vixilancia-libre}
        En este apartado ao igual que no anterior analizamos dous tipos de software, que non están
        intrinsecamente ligados entre si, senón que parece seguir liñas un pouco independentes. Por
        un lado están os programas de vídeo-vixilancia que implementan algún tipo de análise do
        comportamento e por outro lado as aplicacións para investigación científica que permiten 
        detectar comportamento e estudalo mais en detalle.
        
        Do primeiro grupo podemos destacar:
        \begin{itemize}
         \item \textbf{iSpy:}\cite{iSpy-webpage}
            iSpy é un dos software's libres para vídeo-vixilancia máis completos do mercado. Emprega
            unha arquitectura moi semellante á que se precisa neste traballo fin de grao, cun 
            servidor que recolle os sinais de vídeo para logo ofrecelas mediante un acceso web 
            dende calquera navegador. 
            Ao ser un proxecto de código libre sempre é posíbel descargalo para realizar sobre el as
            modificacións relativas á análise do comportamento, mais isto non é preciso posto que 
            dispón da posibilidade de acoplar plugins que modifican ou amplían o seu comportamento.
            Son de especial interese os plugins dispoñibles para realizar tarefas de visión por
            computador\cite{iSpy-plugins}, como poden ser detección e seguimento de actividades,
            recoñecemento de matriculas, conteo de persoas, detección de caras ou análise do 
            comportamento.
         \item \textbf{ZoneMinder:} \cite{zoneMinder-webPage}
            ZoneMinder é xunto con iSpy un dos software's de vídeo-vixilancia libres máis completo
            e potente proporcionando toda unha serie de funcionalidades para amosar vídeo de cámaras
            IP, de circuíto de televisión ou USB. Segue ao igual que iSpy unha arquitectura 
            cliente-servidor, e inda que non ten un abanico de plugins tan rico como o de iSpy 
            dispón de Detección de Movemento integrado\cite{zoneMinder-motion-detection}.
        \end{itemize}
        
        En outra liña diferente, tamén son de interese proxectos de carácter máis investigador que 
        permiten análise do comportamento como son:
        
        \begin{itemize}
         \item \textbf{SwisTrack:}\cite{SwisTrack-webPage}
            SwisTrack é unha ferramenta incriblemente versátil para o seguimento de obxectos, 
            animais, humanos... empregando como fonte de datos unha cámara ou ben un arquivo de 
            vídeo. Distribúese como unha aplicación de escritorio escrita en C++, e que emprega as 
            funcionalidades de OpenCV.
            É de destacar unha opción que permite acceder aos datos da súa análise vía web. O 
            programa dispón dunha interface TCP(Transmission Control Protocol), que amosa en formato
            \underline{text-based NMEA 0183 protocol} o resultado dos compoñentes de saída que 
            escriban nela.
            
         \item \textbf{Community Core Vision:}\cite{ccv-webPage}
            É un proxecto moi similar a SwisTrack, con algo menos de transcendencia pero con 
            soporte multiplataforma, tamén creado como interface de escritorio permite escoller 
            entre 27 opcións de análise diferentes para o tratamento do vídeo.
            
        \item \textbf{Seguimento de Insectos ou partículas:}
            Neste campo tamén son de especial utilidade os programas de tracking, de feito existen 
            dous destacables sistemas para a análise do comportamento dos insectos: BIO-TRACKING
            \cite{bio-tracking-webPage} e Ctrax\cite{ctrax-webPage}, ambos permiten unha detallada
            análise para determinar cales son os comportamentos destes animais en función das
            traxectorias que seguen no seu camiño ou as zoas que máis frecuentan dentro dun
            determinado entorno.
        \end{itemize}
