\chapter{Validación}

Á hora de deseñar probas é importante abarcar a maior parte do código posible, neste proxecto isto
foi todo un reto, pois o alto nivel de integración dificulta enormemente a realización das probas.
Pese a todo, logrouse probar tanto o código realizado en Python-Django así como o código da capa 
cliente en javascript, empregando para elo distintos modelos e bibliotecas de probas que vemos a
continuación.

\section{Probas Unitarias}

    As probas unitarias proban as funcionalidades máis básica do software. Execútanse sempre no 
    ámbito de un só modulo para probar o correcto funcionamento de este, simulando se fose preciso 
    a súa interacción con outros módulos (este proceso chamase mocking).
    
    No ámbito da nosa aplicación as probas de cada módulo recóllense nos ficheiros tests.py de cada
    un deles e nos ficheiros da carpeta ``src/static/site/tests''.
    
    \subsection{Política de acceso ás páxinas web}
        Como resulta lóxico non todo o mundo pode acceder a tódalas páxinas da aplicación, algunhas
        delas están reservadas para o administrador, outras para o propio usuario logueado, unhas
        terceiras para calquera usuario logueado e por último hai páxinas de dominio público.
        
        É importante de cara a non cometer erros de seguridade, que este correcto comportamento sexa
        comprobado, así na táboa excel que se atopa no ficheiro docs/urlsMap.ods pódese observar con
        detalle que política de acceso segue cada unha das páxinas da aplicación segundo a súa URL.
        Foi a partir deste mapa de urls dende onde se elaborou a base dos test de unidade para 
        o acceso ás páxinas, que comprobar para cada unha das páxinas da aplicación que un usuario
        cos permisos adecuados poida acceder e que calquera dos demais reciba o erro axeitado.

    \subsection{Probas da Capa Web con Javascript}
        As probas da capa web escrita en javascript apoiaranse no framework Qunit de 
        jQuery que proporciona un xeito sinxelo de crear probas unitarias sempre e cando o código 
        javascript estea convenientemente separado do HTML que forma a vista da capa web.
        
        Como resulta lóxico, estas probas estarán escritas en javascript e almacenadas no directorio
        do proxecto ``/src/static/site/tests'', podendo executarse de dous xeitos diferentes: Ou ben 
        como unha páxina web pertencente á aplicación (isto favorece o desenvolvemento áxil), ou ben
        como unha proba das realizadas polo comando:
        \begin{verbatim}
        python manage.py test
        \end{verbatim}
        
        Para poder executar un código javascript dende os tests da aplicación, 
        precisase un lanzador ou runner que lance estes tests contra algún navegador de liña de 
        comandos, neste caso a opción seleccionada foi a combinación do paquete django.js (v0.8.1)
        en combinación co navegador de liña de comandos phantomJS. 
        
        Django.js é un conxunto de utilidades que permiten a integración de código javascript en 
        Django. máis en concreto neste proxecto empréganse aquelas que teñen que ver co testing de
        aplicacións \cite{DjangojsTestTools}, destacando as clases QUnitSuite e PhantomJsRunner que se
        empregan para lanzar os tests como parte das probas da aplicación mediante a clase creada
        StaticJsTestCase, obtendo os resultados dos tests mentres que tamén se levanta o servidor
        para atender as posibles peticións do navegador de liña de comandos.
        
        Por outro lado PhantomJS é un navegador WebKit de liña de comandos, cunha API Javascript que
        da soporte rápido e nativo para varios estándares web que resultan moi do noso interese,
        como son a manipulación DOM, os selectores CSS, JSON, Canvas e SVG. PhantomJS será chamado
        implicitamente polo runner de Django.js cando se executen os test, mentres que no caso da 
        vista QUnitView os tests executaranse directamente no navegador que realice a petición.
        
        Pero só con probas unitarias non se pode asegurar a calidade dos software xa que moitos dos
        módulos están pensados para interactuar entre eles, polo que fanse necesarias as Probas de
        Integración.

\section{Probas de Integración}

    As probas de integración so aquelas que proban o funcionamento conxunto de varios módulos da 
    aplicación, realízanse tras o éxito das probas unitarias tamén sen que haxa interacción humana.
    
    No noso caso agruparemos as probas de integración nun paquete a parte, para evitar así 
    mesturalas coas probas unitarias de cada módulo. A estes efectos creamos a clase SeleniumAncowebTest
    que estende StaticLiveServerTestCase engadindo ademais os métodos login\_user(self, user, 
    password) e logout\_user(self, user) xa que todo-los tests que comproben outros módulos precisando
    de un usuario logueado considéranse tests de integración.

    \subsection{Probas Funcionais Selenium}

        As probas funcionais son aquelas nas que se lle dita ao sistema cales serán as saídas a unha 
        determinada serie de entradas co fin de comprobar que a funcionalidade é a correcta. No caso da
        nosa aplicación empregaremos tests funcionais para as probas de integración apoiándonos en Selenium.
        
        Selenium é un framework para a realización de probas funcionais que permite lanzar un navegador
        e indicar as accións a realizar sobre el xunto cunha serie de comprobacións para verificar que
        estas accións provocan na páxina web o efecto desexado. Resultan de especial relevancia na 
        programación web, xa que o seu funcionamento asemellase moito ao que un humano faría para 
        comprobar o correcto funcionamento da web sendo por tanto moi intuitivo.
        
        No noso caso compróbanse con Selenium, o logueado de Usuarios, a subida de vídeos e o listado de
        vídeos.
    
\section{Probas de Sistema}
    As probas de sistema neste caso consisten na comprobación manual de que a páxina se desprega 
    correctamente no entorno de produción. Para esta proba procedese subindo á plataforma mediante 
    o dominio \url{www.ancoweb.es} dous vídeo diferentes: Un representativo de un entorno aberto e público
    e outro pertencente a un espazo interior, posteriormente comprobase que ambos vídeos foron
    subidos e analizados correctamente así como que as distintas compoñentes que interactúan na 
    páxina de reprodución do vídeo funcionan de forma coherente co contido de este. 

\section{Probas de Aceptación}
    Por último están as probas de Aceptación fanse co obxectivo de comprobar se o software cumpre
    coas expectativas que o cliente tiña de el. A estes efectos cada vez que se finalizaba unha 
    funcionalidade realizouse, acorde coa metodoloxía, unha proba completa por parte do titor Brais 
    Cancela que garantise que todo o implementado era acorde co que se desexaba. Ocasionalmente o 
    proxecto tamén foi revisado polos outros titores aportándolle así un carácter máis plural.
    

\section{Calidade}
	Os parámetros de calidade empregados para a codificación do código fonte son os seguintes:
    \begin{itemize}
     \item Para o código \textbf{Javascript}: JavaScript Style Guide and Coding Conventions 
     \cite{javascript-style-guide}, tamén apoiado nas JavaScript Best Practices 
     \cite{javascript-best-practices}.
     \item Con respecto ao código \textbf{Python} seguiuse a convención PEP8 Style Guide for Python
        Code \cite{pepe8-style-guide}.
    \end{itemize}